\documentclass[12pt, letterpaper]{article}
\usepackage{graphicx} % needed for images
\usepackage[export]{adjustbox} % needed for adjustable images
\usepackage{flafter} % needed for figures
\usepackage[table, dvipsnames]{xcolor}
\usepackage{fancyhdr} % needed for fancy header 
\usepackage{listings}
\usepackage{pdflscape}
\usepackage{tabularx}
\usepackage{array}
\usepackage{texlogos}
\usepackage{color}
\usepackage{colortbl}
\usepackage{lineno}
\usepackage{hyperref}
\usepackage{dirtree}

\hypersetup{
    colorlinks=true,
    linkcolor=blue,
    filecolor=magenta,      
    urlcolor=cyan,
    pdfpagemode=FullScreen,
    }
\lstset{aboveskip=20pt,belowskip=20pt}

\lstdefinelanguage{Kotlin}{
  comment=[l]{//},
  commentstyle={\color{gray}\ttfamily},
  emph={filter, first, firstOrNull, forEach, lazy, map, mapNotNull, println},
  emphstyle={\color{OrangeRed}},
  identifierstyle=\color{black},
  keywords={!in, !is, abstract, actual, annotation, as, as?, break, by, catch, class, companion, const, constructor, continue, crossinline, data, delegate, do, dynamic, else, enum, expect, external, false, field, file, final, finally, for, fun, get, if, import, in, infix, init, inline, inner, interface, internal, is, lateinit, noinline, null, object, open, operator, out, override, package, param, private, property, protected, public, receiveris, reified, return, return@, sealed, set, setparam, super, suspend, tailrec, this, throw, true, try, typealias, typeof, val, var, vararg, when, where, while},
  keywordstyle={\color{NavyBlue}\bfseries},
  morecomment=[s]{/*}{*/},
  morestring=[b]",
  morestring=[s]{"""*}{*"""},
  ndkeywords={@Deprecated, @JvmField, @JvmName, @JvmOverloads, @JvmStatic, @JvmSynthetic, Array, Byte, Double, Float, Int, Integer, Iterable, Long, Runnable, Short, String, Any, Unit, Nothing},
  ndkeywordstyle={\color{BurntOrange}\bfseries},
  sensitive=true,
  stringstyle={\color{ForestGreen}\ttfamily},
}

\graphicspath{ {images/} }

\pagestyle{fancy}
\fancyhf{}
\fancyhead[LE,RO]{Aplicacions per a dispositius mòbils}
\fancyhead[RE,LO]{Resources en Android}
\fancyfoot[LE,RO]{\thepage}
\renewcommand{\headrulewidth}{1pt}
\renewcommand{\footrulewidth}{1pt}



% information
\title{%
    \begin{center}
	\includegraphics[width=4cm,height=3cm]{udl.png}
    \end{center}
    \line(1,0){250}\\[0.3cm]
    \textbf{Primera Activitat: Resources en Android} \\
    \line(1,0){250}
    \\[0.5cm]
	\large Aplicacions per a dispositius mòbils - Grau en Enginyería Informàtica
}
\author{Pablo Fraile Alonso}
\date{\today}

% document
\begin{document}
    
% title
\maketitle
\thispagestyle{empty}
\newpage
\tableofcontents
\listoffigures
\listoftables
\newpage

\section{Comprovar si s'aplica la primera bona pràctica}
Veiem l'estructura del projecte i en un principi si que compleix la primera bona pràctica, ja que té diferents recursos separats en diferents arxius, tal i com es veu en l'arbre d'arxius:

\dirtree{%
.1 FirstActivity.
.2 app.
.3 {...} .
.3 src.
.4 {...} .
.4 main.
.5 AndroidManifest.xml.
.5 java.
.5 res.
.6 drawable-{...}.
.6 layout.
.6 mipmap-{....}.
.6 values.
}

En cambi, si ens fixem al arxiu de layout, veiem que tot i que tingui creat l'arxiu \textit{strings.xml}, han "hardcodejat" la string "Hello World" dins de la component Textview:
\begin{verbatim}
<TextView
    android:id="@+id/textView"
    ....
    android:text="Hello World"
    ...
/>
\end{verbatim}

Quant realment, si es volgués separar els diferents recursos, s'hauria de cambiar el text a que referencii a les strings localitzades a: res/values/strings.xml.

\begin{verbatim}
<TextView
    android:id="@+id/textView"
    ....
    android:text="@string/hello_world"
    ...
/>
\end{verbatim}

\section{Comprobar si se aplica la segunda buena práctica y añadir recursos alternativos para la app. }
La segona pràctica consisteix en "Provide alternative resources to support specific device configurations". Veiem que tot i que Android Studio ens hagi proporcionat diferents resources, aquests únicament venen adaptat
per un dispositiu móvil (no tablet), amb layout portrait (vertical) i idioma anglés, per tant es podría dir que no compleix la segona bona pràctica.


\dirtree{%
.0 res.
.1 drawable.
.2 image\_1.png.
.2 image\_2.png.
.2 image\_3.png.
.1 drawable-ca.
.2 image\_1.png.
.2 image\_2.png.
.2 image\_3.png.
.1 drawable-es.
.2 image\_1.png.
.2 image\_2.png.
.2 image\_3.png.
.1 layout.
.2 activity\_main.xml.
.1 layout-land.
.2 activity\_main.xml.
.1 layout-large.
.2 activity\_main.xml.
.1 layout-large-port.
.2 activity\_main.xml.
.1 mipmap-anydpi-v26.
.2 ic\_launcher\_round.xml.
.2 ic\_launcher.xml.
.1 mipmap-ca-hdpi.
.2 ic\_launcher\_foreground.png.
.2 ic\_launcher.png.
.2 ic\_launcher\_round.png.
.1 mipmap-ca-mdpi.
.2 ic\_launcher\_foreground.png.
.2 ic\_launcher.png.
.2 ic\_launcher\_round.png.
.1 mipmap-ca-xhdpi.
.2 ic\_launcher\_foreground.png.
.2 ic\_launcher.png.
.2 ic\_launcher\_round.png.
.1 mipmap-ca-xxhdpi.
.2 ic\_launcher\_foreground.png.
.2 ic\_launcher.png.
.2 ic\_launcher\_round.png.
.1 mipmap-ca-xxxhdpi.
.2 ic\_launcher\_foreground.png.
.2 ic\_launcher.png.
.2 ic\_launcher\_round.png.
.1 mipmap-es-hdpi.
.2 ic\_launcher\_foreground.png.
.2 ic\_launcher.png.
.2 ic\_launcher\_round.png.
.1 mipmap-es-mdpi.
.2 ic\_launcher\_foreground.png.
.2 ic\_launcher.png.
.2 ic\_launcher\_round.png.
.1 mipmap-es-xhdpi.
.2 ic\_launcher\_foreground.png.
.2 ic\_launcher.png.
.2 ic\_launcher\_round.png.
.1 mipmap-es-xxhdpi.
.2 ic\_launcher\_foreground.png.
.2 ic\_launcher.png.
.2 ic\_launcher\_round.png.
.1 mipmap-es-xxxhdpi.
.2 ic\_launcher\_foreground.png.
.2 ic\_launcher.png.
.2 ic\_launcher\_round.png.
.1 mipmap-hdpi.
.2 ic\_launcher\_foreground.png.
.2 ic\_launcher.png.
.2 ic\_launcher\_round.png.
.1 mipmap-mdpi.
.2 ic\_launcher\_foreground.png.
.2 ic\_launcher.png.
.2 ic\_launcher\_round.png.
.1 mipmap-xhdpi.
.2 ic\_launcher\_foreground.png.
.2 ic\_launcher.png.
.2 ic\_launcher\_round.png.
.1 mipmap-xxhdpi.
.2 ic\_launcher\_foreground.png.
.2 ic\_launcher.png.
.2 ic\_launcher\_round.png.
.1 mipmap-xxxhdpi.
.2 ic\_launcher\_foreground.png.
.2 ic\_launcher.png.
.2 ic\_launcher\_round.png.
.1 values.
.2 colors.xml.
.2 ic\_launcher\_background.xml.
.2 strings.xml.
.2 themes.xml.
.1 values-ca.
.2 ic\_launcher\_background.xml.
.2 strings.xml.
.1 values-es-rES.
.2 ic\_launcher\_background.xml.
.2 strings.xml.
.1 values-night.
.2 themes.xml.
}

\section{Añadir la funcionalidad de mostrar un mensaje emergente (Toast) en un botón del layout}
Se han provado varias opciones:
\subsection{Pasar como parametro a la función setOnClickListener una View con un método implementado onClickListener}
\begin{lstlisting}[language=Kotlin]
    val button = findViewById<Button>(R.id.button)
    button.setOnClickListener (
        View.OnClickListener() {
            Toast.makeText(
                applicationContext,
                R.string.toastText,
                Toast.LENGTH_SHORT
            ).show()
        })
    }
\end{lstlisting}
Que en kotlin, con su estilo de lambda, podemos simplificar a:

\begin{lstlisting}[language=Kotlin]
    val button = findViewById<Button>(R.id.button)
    button.setOnClickListener {
        Toast.makeText(
            applicationContext,
            R.string.toastText,
            Toast.LENGTH_SHORT
        ).show()
    }
\end{lstlisting}

También, podriamos haver creado una inner class Toaster para poder ejecutar el boton:
\begin{lstlisting}[language=Kotlin]
    private fun setUpButton() {
        val button = findViewById<Button>(R.id.button)
        button.setOnClickListener (Toaster())
    }

    private inner class Toaster : View.OnClickListener {
        override fun onClick(p0: View?) {
            Toast.makeText(
                applicationContext,
                R.string.toastText,
                Toast.LENGTH_LONG
            ).show()
        }
    }
\end{lstlisting}

En caso de que la hiciesemos estática, tendriamos un poco más de problemas, ya que deveríamos passar la View por parámetro para poder obtener el contexto:
\begin{lstlisting}[language=Kotlin]
    fun setUpButton() {
        val button = findViewById<Button>(R.id.button)
        button.setOnClickListener (Toaster())
    }

    private class Toaster : View.OnClickListener {
        override fun onClick(p0: View?) {
            if (p0 != null) {
                Toast.makeText(
                    p0.context,
                    R.string.toastText,
                    Toast.LENGTH_SHORT
                ).show()
            }
        }
    }
\end{lstlisting}


\end{document}
